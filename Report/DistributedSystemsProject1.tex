%	PACKAGES AND OTHER DOCUMENT CONFIGURATIONS
%---------------------------------------------------------------------------------------
\documentclass{article}

\usepackage{fancyhdr} % Required for custom headers
\usepackage{lastpage} % Required to determine the last page for the footer
\usepackage{extramarks} % Required for headers and footers
\usepackage{graphicx} % Required to insert images
\usepackage{amssymb}
\usepackage{amsmath}
\usepackage{multicol}
\usepackage{listings}
%\usepackage{apacite}
\usepackage{lipsum}
\usepackage[nottoc,notlot,notlof]{tocbibind}
\usepackage{color} %red, green, blue, yellow, cyan, magenta, black, white
\usepackage{xcolor}
\usepackage{array}
\newcolumntype{P}[1]{>{\centering\arraybackslash}p{#1}}
\usepackage{hyperref}
\hypersetup{
    colorlinks,
    linkcolor={blue!80!black}, %red!50!black
    citecolor={blue!50!black},
    urlcolor={blue!80!black}
}
\usepackage{float}
\usepackage{enumitem}
\usepackage{titlesec}
\usepackage{verbatim}
\usepackage{tocloft}
\setlength{\cftsecnumwidth}{3em}
\setlength{\cftsubsecnumwidth}{4em}
\setlength{\cftsubsubsecnumwidth}{5em}
\lstset{language=Matlab}
\lstset{flexiblecolumns=true}
\definecolor{mygreen}{RGB}{28,172,0} % color values Red, Green, Blue
\definecolor{mylilas}{RGB}{170,55,241}

% Margins
\topmargin=-0.45in
\evensidemargin=0in
\oddsidemargin=0in
\textwidth=6.5in
\textheight=9.0in
\headsep=0.25in 

\linespread{1.1} % Line spacing

% Set up the header and footer
\pagestyle{fancy}
\lhead{Sabbir Rashid \& Rob Berman} % Top left header
\rhead{\textit{\ProjectTitle}} % Top center header
%\rhead{\firstxmark} % Top right header
%\lfoot{\lastxmark} % Bottom left footer
\lfoot{\Institution}
\cfoot{} % Bottom center footer
\rfoot{Page\ \thepage\ of\ \pageref{LastPage}} % Bottom right footer
\renewcommand\headrulewidth{0.4pt} % Size of the header rule
\renewcommand\footrulewidth{0.4pt} % Size of the footer rule

\setlength\parindent{0pt} % Removes all indentation from paragraphs

% Title and Authors
\newcommand{\ProjectTitle}{Raymonds Algorithm - A Centralized Approach} % Assignment title
\newcommand{\Institution}{Renssselaer Polytechnic Institute}

%% Section formating
\titleclass{\subsubsubsection}{straight}[\subsection]

\newcounter{subsubsubsection}[subsubsection]

\renewcommand{\thesection}{\Roman{section}} 
\renewcommand{\thesubsection}{\thesection.\Roman{subsection}}
\renewcommand{\thesubsubsection}{\thesubsection.\roman{subsubsection}}

\renewcommand\thesubsubsubsection{\thesubsubsection.\roman{subsubsubsection}}
\renewcommand\theparagraph{\thesubsubsubsection.\arabic{paragraph}} % optional; useful if paragraphs are to be numbered

\titleformat{\section}[block]
  {\fontsize{14}{15}\bfseries\sffamily\filcenter}
  {\thesection}
  {1em}
  {\MakeUppercase}
  
\titleformat{\subsection}[block]
  {\fontsize{12}{15}\bfseries\sffamily}
  {\thesubsection}{1em}  {}

\titleformat{\subsubsection}[block]
  {\normalfont\normalsize\bfseries}{\thesubsubsection}{1em}{}
  
\titleformat{\subsubsubsection}[block]
  {\normalfont\normalsize\bfseries}{\thesubsubsubsection}{1em}{}
  
\titlespacing*{\subsubsubsection}
{0pt}{3.25ex plus 1ex minus .2ex}{1.5ex plus .2ex}

\makeatletter
\renewcommand\paragraph{\@startsection{paragraph}{5}{\z@}%
  {3.25ex \@plus1ex \@minus.2ex}%
  {-1em}%
  {\normalfont\normalsize\bfseries}}
\renewcommand\subparagraph{\@startsection{subparagraph}{6}{\parindent}%
  {3.25ex \@plus1ex \@minus .2ex}%
  {-1em}%
  {\normalfont\normalsize\bfseries}}
\def\toclevel@subsubsubsection{4}
\def\toclevel@paragraph{5}
\def\toclevel@paragraph{6}
\def\l@subsubsubsection{\@dottedtocline{4}{7em}{4em}}
\def\l@paragraph{\@dottedtocline{5}{10em}{5em}}
\def\l@subparagraph{\@dottedtocline{6}{14em}{6em}}
\makeatother
\setcounter{secnumdepth}{4}
\setcounter{tocdepth}{4}


%	TITLE PAGE
%----------------------------------------------------------------------------------------

\title{
%\vspace{2in}
\textmd{\textbf{\ProjectTitle}}\\
\author{
	Sabbir Rashid \& Rob Berman\\
	\bf{\Institution}}
	\date{} %
}

\begin{document}
\lstset{language=Java,%
    %basicstyle=\color{red},
    breaklines=true,%
    morekeywords={matlab2tikz},
    keywordstyle=\color{blue},%
    morekeywords=[2]{1}, keywordstyle=[2]{\color{black}},
    identifierstyle=\color{black},%
    stringstyle=\color{mylilas},
    commentstyle=\color{mygreen},%
    showstringspaces=false,%without this there will be a symbol in the places where there is a space
    numbers=left,%
    numberstyle={\tiny \color{black}},% size of the numbers
    numbersep=9pt, % this defines how far the numbers are from the text
    emph=[1]{for,end,break},emphstyle=[1]\color{red}, %some words to emphasise
    %emph=[2]{word1,word2}, emphstyle=[2]{style},    
    numbers=none,
}

\setlength{\parindent}{8ex}
%----------------------------------------------------------------------------------------
%	TABLE OF CONTENTS
%----------------------------------------------------------------------------------------

\tableofcontents
\newpage
\maketitle
\begin{multicols}{2}
%------------------------------------------------------------------------------------------
\section{Abstract}
\label{sec:Abstract}
\textit{}
%----------------------------------------------------------------------------------------
%	INTRODUCTION
%----------------------------------------------------------------------------------------
\section{Introduction}
\label{sec:Introduction}
\lipsum[1]
	\subsection{Problem Statement}
	\label{subsec:ProblemStatement}
	\lipsum[2]

\section{Methodology}
\label{sec:Methodology}
\lipsum[1]
	\subsection{Raymond's Functions}
	\label{subsec:RaymondsFunctions}
	\lipsum[2]
		\subsubsection{assignToken(Process p)}
		\begin{lstlisting}
public void assignToken(Process p) {
  if ( (p.holderEnum == Process.HolderEnum.Self) && (!p.usingResource) && (!p.requestQueue.isEmpty()) ) {
    holderProc = p.requestQueue.pop() ;

    if (p.getProcessID() == holderProc.getProcessID()) { //i.e. the process p is at the front of its own queue
      p.holderEnum = Process.HolderEnum.Self;
    } else {
      p.holderEnum = Process.HolderEnum.Neighbor;
      holderProc.holderEnum = Process.HolderEnum.Self ;
    }
	
    p.asked = false;
	
    if (p.holderEnum == Process.HolderEnum.Self) {
      p.usingResource = true;
    } else {
      assignToken(holderProc); // Check this, supposed to be "send token to holder"
    }
  }
}
		\end{lstlisting}
		
		\subsubsection{sendRequest(Process p)}
		\begin{lstlisting}
public void sendRequest(Process p) {
  if ( (p.holderEnum != Process.HolderEnum.Self) && (!p.requestQueue.isEmpty()) && (!p.asked) ) {
    sendRequest(holderProc);
    p.asked = true;
  }
}
		\end{lstlisting}
		\subsubsection{requestResource(Process p)}
		\begin{lstlisting}
public void requestResource(Process p) {
  p.requestQueue.push(p);
  assignToken(p);
  sendRequest(p);
}
		\end{lstlisting}
		
		\subsubsection{releaseResource(Process p)}
		\begin{lstlisting}
public void releaseResource(Process p) {
  p.usingResource = false;
  assignToken(p);
  sendRequest(p);
}
		\end{lstlisting}
		
		\subsubsection{receivedRequestFromNeighbor(Process p, Process neighbor)}
		\begin{lstlisting}
public void receivedRequestFromNeighbor(Process p, Process neighbor) {
  p.requestQueue.push(neighbor);
  assignToken(p);
  sendRequest(p);
}
		\end{lstlisting}
		
		\subsubsection{receivedToken(Process p)}
		\begin{lstlisting}
public void receivedToken(Process p) {
  p.holderEnum = Process.HolderEnum.Self ;
  holderProc = p;
  assignToken(p);
  sendRequest(p);		
}
		\end{lstlisting}
\section{Conclusions}	
\lipsum[42]

%\newpage
\end{multicols}
\end{document}