%	PACKAGES AND OTHER DOCUMENT CONFIGURATIONS
%---------------------------------------------------------------------------------------
\documentclass{article}

\usepackage{fancyhdr} % Required for custom headers
\usepackage{lastpage} % Required to determine the last page for the footer
\usepackage{extramarks} % Required for headers and footers
\usepackage{graphicx} % Required to insert images
\usepackage{amssymb}
\usepackage{amsmath}
\usepackage{multicol}
\usepackage{listings}
%\usepackage{apacite}
\usepackage{lipsum}
\usepackage[nottoc,notlot,notlof]{tocbibind}
\usepackage{color} %red, green, blue, yellow, cyan, magenta, black, white
\usepackage{xcolor}
\usepackage{array}
\newcolumntype{P}[1]{>{\centering\arraybackslash}p{#1}}
\usepackage{hyperref}
\hypersetup{
    colorlinks,
    linkcolor={blue!80!black}, %red!50!black
    citecolor={blue!50!black},
    urlcolor={blue!80!black}
}
\usepackage{float}
\usepackage{enumitem}
\usepackage{titlesec}
\usepackage{verbatim}
\usepackage{tocloft}
\setlength{\cftsecnumwidth}{3em}
\setlength{\cftsubsecnumwidth}{4em}
\setlength{\cftsubsubsecnumwidth}{5em}
\lstset{language=Matlab}
\lstset{flexiblecolumns=true}
\definecolor{mygreen}{RGB}{28,172,0} % color values Red, Green, Blue
\definecolor{mylilas}{RGB}{170,55,241}

% Margins
\topmargin=-0.45in
\evensidemargin=0in
\oddsidemargin=0in
\textwidth=6.5in
\textheight=9.0in
\headsep=0.25in 

\linespread{1.1} % Line spacing

% Set up the header and footer
\pagestyle{fancy}
\lhead{Sabbir Rashid \& Rob Berman} % Top left header
\rhead{\textit{\ProjectTitle}} % Top center header
%\rhead{\firstxmark} % Top right header
%\lfoot{\lastxmark} % Bottom left footer
\lfoot{\Institution}
\cfoot{} % Bottom center footer
\rfoot{Page\ \thepage\ of\ \pageref{LastPage}} % Bottom right footer
\renewcommand\headrulewidth{0.4pt} % Size of the header rule
\renewcommand\footrulewidth{0.4pt} % Size of the footer rule

\setlength\parindent{0pt} % Removes all indentation from paragraphs

% Title and Authors
\newcommand{\ProjectTitle}{Raymonds Algorithm - A Centralized Approach} % Assignment title
\newcommand{\Institution}{Renssselaer Polytechnic Institute}

%% Section formating
\titleclass{\subsubsubsection}{straight}[\subsection]

\newcounter{subsubsubsection}[subsubsection]

\renewcommand{\thesection}{\Roman{section}} 
\renewcommand{\thesubsection}{\thesection.\Roman{subsection}}
\renewcommand{\thesubsubsection}{\thesubsection.\roman{subsubsection}}

\renewcommand\thesubsubsubsection{\thesubsubsection.\roman{subsubsubsection}}
\renewcommand\theparagraph{\thesubsubsubsection.\arabic{paragraph}} % optional; useful if paragraphs are to be numbered

\titleformat{\section}[block]
  {\fontsize{14}{15}\bfseries\sffamily\filcenter}
  {\thesection}
  {1em}
  {\MakeUppercase}
  
\titleformat{\subsection}[block]
  {\fontsize{12}{15}\bfseries\sffamily}
  {\thesubsection}{1em}  {}

\titleformat{\subsubsection}[block]
  {\normalfont\normalsize\bfseries}{\thesubsubsection}{1em}{}
  
\titleformat{\subsubsubsection}[block]
  {\normalfont\normalsize\bfseries}{\thesubsubsubsection}{1em}{}
  
\titlespacing*{\subsubsubsection}
{0pt}{3.25ex plus 1ex minus .2ex}{1.5ex plus .2ex}

\makeatletter
\renewcommand\paragraph{\@startsection{paragraph}{5}{\z@}%
  {3.25ex \@plus1ex \@minus.2ex}%
  {-1em}%
  {\normalfont\normalsize\bfseries}}
\renewcommand\subparagraph{\@startsection{subparagraph}{6}{\parindent}%
  {3.25ex \@plus1ex \@minus .2ex}%
  {-1em}%
  {\normalfont\normalsize\bfseries}}
\def\toclevel@subsubsubsection{4}
\def\toclevel@paragraph{5}
\def\toclevel@paragraph{6}
\def\l@subsubsubsection{\@dottedtocline{4}{7em}{4em}}
\def\l@paragraph{\@dottedtocline{5}{10em}{5em}}
\def\l@subparagraph{\@dottedtocline{6}{14em}{6em}}
\makeatother
\setcounter{secnumdepth}{4}
\setcounter{tocdepth}{4}


%	TITLE PAGE
%----------------------------------------------------------------------------------------

\title{
%\vspace{2in}
\textmd{\textbf{\ProjectTitle}}\\
\author{
	Sabbir Rashid \& Rob Berman\\
	\bf{\Institution}}
	\date{} %
}

\begin{document}
\lstset{language=Java,%
    %basicstyle=\color{red},
    breaklines=true,%
    morekeywords={matlab2tikz},
    keywordstyle=\color{blue},%
    morekeywords=[2]{1}, keywordstyle=[2]{\color{black}},
    identifierstyle=\color{black},%
    stringstyle=\color{mylilas},
    commentstyle=\color{mygreen},%
    showstringspaces=false,%without this there will be a symbol in the places where there is a space
    numbers=left,%
    numberstyle={\tiny \color{black}},% size of the numbers
    numbersep=9pt, % this defines how far the numbers are from the text
    emph=[1]{for,end,break},emphstyle=[1]\color{red}, %some words to emphasise
    %emph=[2]{word1,word2}, emphstyle=[2]{style},    
    numbers=none,
}

\setlength{\parindent}{8ex}
%----------------------------------------------------------------------------------------
%	TABLE OF CONTENTS
%----------------------------------------------------------------------------------------

\tableofcontents
\newpage
\maketitle
%\begin{multicols}{2}
%------------------------------------------------------------------------------------------
\section{Abstract}
\label{sec:Abstract}
\textit{}
%----------------------------------------------------------------------------------------
%	INTRODUCTION
%----------------------------------------------------------------------------------------
\section{Introduction}
\label{sec:Introduction}
\lipsum[1]
	\subsection{Problem Statement}
	\label{subsec:ProblemStatement}
	\lipsum[2]

\section{Methodology}
\label{sec:Methodology}
\lipsum[1]
	\subsection{Raymond's Functions}
	\label{subsec:RaymondsFunctions}
	\lipsum[2]
		\subsubsection{assignToken(Process p)}
		\begin{lstlisting}
public void assignToken(Process p) {
  if ( (p.holderEnum == Process.HolderEnum.Self) && (!p.usingResource) && (!p.requestQueue.isEmpty()) ) {
    holderProc = p.requestQueue.pop() ;

    if (p.getProcessID() == holderProc.getProcessID()) { //i.e. the process p is at the front of its own queue
      p.holderEnum = Process.HolderEnum.Self;
    } else {
      p.holderEnum = Process.HolderEnum.Neighbor;
      holderProc.holderEnum = Process.HolderEnum.Self ;
    }
	
    p.asked = false;
	
    if (p.holderEnum == Process.HolderEnum.Self) {
      p.usingResource = true;
    } else {
      assignToken(holderProc); // Check this, supposed to be "send token to holder"
    }
  }
}
		\end{lstlisting}
		
		\subsubsection{sendRequest(Process p)}
		\begin{lstlisting}
public void sendRequest(Process p) {
  if ( (p.holderEnum != Process.HolderEnum.Self) && (!p.requestQueue.isEmpty()) && (!p.asked) ) {
    sendRequest(holderProc);
    p.asked = true;
  }
}
		\end{lstlisting}
		\subsubsection{requestResource(Process p)}
		\begin{lstlisting}
public void requestResource(Process p) {
  p.requestQueue.push(p);
  assignToken(p);
  sendRequest(p);
}
		\end{lstlisting}
		
		\subsubsection{releaseResource(Process p)}
		\begin{lstlisting}
public void releaseResource(Process p) {
  p.usingResource = false;
  assignToken(p);
  sendRequest(p);
}
		\end{lstlisting}
		
		\subsubsection{receivedRequestFromNeighbor(Process p, Process neighbor)}
		\begin{lstlisting}
public void receivedRequestFromNeighbor(Process p, Process neighbor) {
  p.requestQueue.push(neighbor);
  assignToken(p);
  sendRequest(p);
}
		\end{lstlisting}
		
		\subsubsection{receivedToken(Process p)}
		\begin{lstlisting}
public void receivedToken(Process p) {
  p.holderEnum = Process.HolderEnum.Self ;
  holderProc = p;
  assignToken(p);
  sendRequest(p);		
}
		\end{lstlisting}
	\subsection{Client Code}
		\begin{lstlisting}
package sockets;

import java.io.*;
import java.net.*;


public class Client {
	public static void main(String[] args) throws IOException {
        
        if (args.length != 2) {
            System.err.println(
                "Usage: java Client <host name> <port number>");
            System.exit(1);
        }
 
        String hostName = args[0];
        int portNumber = Integer.parseInt(args[1]);
        System.out.println("CLIENT: About to try to create Client Socket");
        try (
            Socket clientSocket = new Socket(hostName, portNumber);
        	PrintWriter out =
                new PrintWriter(clientSocket.getOutputStream(), true);
        	BufferedReader in =
                new BufferedReader(
                    new InputStreamReader(clientSocket.getInputStream()));
            BufferedReader stdIn =
                new BufferedReader(
                    new InputStreamReader(System.in));
        	){
        	String userInput;
        	System.out.println("CLIENT: About to wait for user input.");
        	System.out.println("Select the following command that you want to execute:");
    		System.out.println("1: create <filename>: creates an empty file named <filename>");
    		System.out.println("2: delete <filename>: deletes file named <filename>");
    		System.out.println("3: read <filename>: displays the contents of <filename>");
    		System.out.println("4: append <filename> <line>: appends a <line> to <filename>");
    		System.out.println("5: exit: exits the program");
            
    		while ((userInput = stdIn.readLine()) != null) {
    			out.println(userInput);
    			System.out.println(in.readLine());
    			if (userInput.contains("read")){
    				String ans = "";
    				while(in.ready())
                    {
    					ans=in.readLine();
                    	System.out.println("CLIENT: In inner while loop.");
                    	System.out.println(ans);
                    }
                    System.out.println("CLIENT: Exited inner while loop.");
    			}
    		}
            System.out.println("CLIENT: Exited while loop.");
            clientSocket.close();
        } catch (UnknownHostException e) {
            System.err.println("Don't know about host " + hostName);
            System.exit(1);
        } catch (IOException e) {
            System.err.println("Couldn't get I/O for the connection to " +
                hostName);
            System.exit(1);
        } 
    }
}
		\end{lstlisting}
	\subsection{Multithreaded Server Code}
		\begin{lstlisting}
package sockets;

import java.io.BufferedReader;
import java.io.File;
import java.io.InputStreamReader;
import java.io.OutputStreamWriter;
import java.io.PrintWriter;
import java.net.ServerSocket;
import java.net.Socket;

import main.Main;

public class MultiThread {
	public static void main(String[] args) throws Exception {
		if (args.length != 1) {
			System.err.println("Usage: java Server <port number>");
			System.exit(1);
		}

		int portNumber = Integer.parseInt(args[0]);
		
		@SuppressWarnings("resource")
		ServerSocket m_ServerSocket = new ServerSocket(portNumber);
		
		int id = 0;
		while (true) {
			Socket clientSocket = m_ServerSocket.accept();
			ClientServiceThread cliThread = new ClientServiceThread(clientSocket, id++);
			cliThread.start();
		}
	}
}

class ClientServiceThread extends Thread {
	Socket clientSocket;
	int clientID = -1;
	boolean running = true;

	ClientServiceThread(Socket s, int i) {
		clientSocket = s;
		clientID = i;
	}

	public void run() {
		System.out.println("Accepted Client : ID - " + clientID + " : Address - "
				+ clientSocket.getInetAddress().getHostName());
		try {
			BufferedReader   in = new BufferedReader(new InputStreamReader(clientSocket.getInputStream()));
			System.out.println("SERVER: Created buffered reader in.");
			PrintWriter   out = new PrintWriter(new OutputStreamWriter(clientSocket.getOutputStream()),true);
			System.out.println("SERVER: Created print writer out.");
            
			while (running) {	
				System.out.println("SERVER: In running loop.");
				//String result = console.nextLine();
				//String result = inputLine;
				String result = in.readLine();
				//Note: Calling create, delete, read, and append go here:
				File testFile = null;
				if(result.substring(0,6).equalsIgnoreCase("create"))
				{
					out.println("Creating File...");
					testFile = Main.CreateFile(result.substring(7,result.length()));
				}
				else if(result.substring(0,6).equalsIgnoreCase("delete"))
				{
					out.println("Deleting File...");
					Main.DeleteFile(result.substring(7,result.length()));
				}
				else if(result.substring(0,4).equalsIgnoreCase("read"))
				{
					String temp = Main.ReadFile(result.substring(5,result.length()));
					out.println("Reading File...\n" + temp);
					out.flush();
				}
				else if(result.substring(0,6).equalsIgnoreCase("append"))
				{
					out.println("Appending to File...");
					String tmp = result.substring(7,result.length());
					int index = tmp.indexOf(' ');
					Main.AppendFile(tmp.substring(0,index),tmp.substring(index+1,tmp.length()));
				}
				else if(result.substring(0,4).equalsIgnoreCase("exit"))
				{
					out.println("Exiting...");
					out.flush();
					running = false;
					System.out.print("Stopping client thread for client : " + clientID);
					Main.ExitConnection();
				}
				else
					out.println("Error: Invalid Command");

			}
		} catch (Exception e) {
			e.printStackTrace();
		}
	}
}
		\end{lstlisting}
		
		\section{Main - Read, Write, Append, Delete}
		\begin{lstlisting}
package main;

import java.io.BufferedReader;
import java.io.BufferedWriter;
import java.io.File;
import java.io.FileReader;
import java.io.FileWriter;
import java.io.IOException;
import java.util.ArrayList;
import java.util.Scanner;
import raymonds.Process;

public class Main {

	public static void main(String[] args) throws IOException {
		FileReader fr = new FileReader("tree.txt");
		String input = br.readLine();
		boolean first = true;
		ArrayList<Process> processes = new ArrayList<Process>();
		while(input!=null)
		{
			if(first)
			{
				processes.add(new Process(input.substring(1, 2),Process.HolderEnum.Neighbor,false,false));
				processes.get(0).addNeighbor(new Process(input.substring(3, 4),Process.HolderEnum.Neighbor,false,false));
				processes.add(new Process(input.substring(3, 4),Process.HolderEnum.Neighbor,false,false));
				processes.get(1).addNeighbor(new Process(input.substring(1, 2),Process.HolderEnum.Neighbor,false,false));
				first = false;
			}
			else
			{
				int index = 0;
				boolean found=false;
				for(int i=0;i<processes.size();i++)
				{
					if(input.substring(1,2).equals(processes.get(i).getProcessID()))
					{
						found=true;
						index=i;
					}
					
				}
				if(!found)
				{
					processes.add(new Process(input.substring(1, 2),Process.HolderEnum.Neighbor,false,false));
					processes.get(processes.size()-1).addNeighbor(new Process(input.substring(3, 4),Process.HolderEnum.Neighbor,false,false));
				}
				else
				{
					if(!processes.get(index).getNeighbors().contains(new Process(input.substring(3, 4),Process.HolderEnum.Neighbor,false,false)))
					{
						processes.get(index).addNeighbor(new Process(input.substring(3, 4),Process.HolderEnum.Neighbor,false,false));
						System.out.println("CONTAINS 1");
					}
				}
				found=false;
				index = 0;
				for(int i=0;i<processes.size();i++)
				{
					if(input.substring(3,4).equals(processes.get(i).getProcessID()))
					{
						found=true;
						index=i;
					}
				}
				if(!found)
				{
					processes.add(new Process(input.substring(3, 4),Process.HolderEnum.Neighbor,false,false));
					processes.get(processes.size()-1).addNeighbor(new Process(input.substring(1, 2),Process.HolderEnum.Neighbor,false,false));
				}
				else
				{
					if(!processes.get(index).getNeighbors().contains(new Process(input.substring(1, 2),Process.HolderEnum.Neighbor,false,false)))
					{
						processes.get(index).addNeighbor(new Process(input.substring(1, 2),Process.HolderEnum.Neighbor,false,false));
						System.out.println("CONTAINS 2");
					}
				}
			}
			input=br.readLine();
		}
		for(int i=0;i<processes.size();i++)
		{
			System.out.println(processes.get(i).getProcessID());
			System.out.print("Neighbors: ");
			for(int j=0;j<processes.get(i).getNeighbors().size();j++)
				System.out.print(processes.get(i).getNeighbors().get(j).getProcessID()+" ");
			System.out.println();
		}
	}
	
	public static File CreateFile( String fileName) throws IOException {
		
		File file = new File(fileName);
		file.createNewFile();
		return file;
	}

	public static void AppendFile( String fileName, String line) throws IOException {

		FileWriter writer = new FileWriter(fileName, true);
		writer.append(line + "\n");
		writer.flush();
		writer.close();
		
	}
	
	public static String ReadFile( String fileName) throws IOException {
		BufferedReader reader = new BufferedReader(new FileReader(fileName));
		String input = reader.readLine();
		String result = input;
		while(input!=null)
		{
			input=reader.readLine();
			if(input!=null)
				result = result + "\n"+ input;
		}
		reader.close();
		return result;
	}

	public static void DeleteFile( String fileName) throws IOException {
		Runtime.getRuntime().exec(new String[]{"bash","-c","rm " + fileName});
	}
	
	public static void ExitConnection()
	{
		
	}

}
		\end{lstlisting}
\section{Conclusions}	
\lipsum[42]

%\newpage
%\end{multicols}
\end{document}